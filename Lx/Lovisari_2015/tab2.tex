\begin{table*}[htp]
    \caption{Derived properties for the galaxy groups.}
    $$
    \centering
    \setlength\extrarowheight{2pt}
    \begin{tabular}{ccccccccccc}
    {\rm Group \ name}  & kT & R_{spec} & R_{500} & M_{500} & M_{gas,500} & R_{2500}  & M_{2500} & M_{gas,2500} & t_{cool} & L_{X,xmm}\\ 
                       & {\rm kev} & {\rm h^{-1}_{70} \ kpc} & {\rm h^{-1}_{70} \ kpc}  & {\rm 10^{13} \ h^{-1}_{70} \ M_{\sun} } & {\rm 10^{12} \ h^{-5/2}_{70} \ M_{\sun}} & {\rm h^{-1}_{70} \ kpc} & {\rm 10^{13} \ h^{-1}_{70} \ M_{\odot} } & {\rm 10^{12} \ h^{-5/2}_{70}} & {\rm Gyr} & {\rm 10^{43} \ erg \ s^{-1}} \\ 
    % \noalign{\smallskip}
    % \hline
    {\rm NGC4936} & 0.85\pm0.03 & 167 & 417\pm21 & 2.07\pm0.32 & 1.69\pm0.24 & 187\pm9 & 0.93\pm0.14 & 0.49\pm0.02 & 0.69\pm0.15 & 0.42\pm0.04 \\ 
    {\rm S0753}   & 1.51\pm0.03 & 219 & 547\pm36 & 4.67\pm0.94 & 3.53\pm0.19 & 211\pm5 & 1.33\pm0.11 & 0.62\pm0.04 & 1.30\pm0.26 & 0.69\pm0.08  \\ 
    {\rm HCG62}   & 1.05\pm0.01 & 245 & 437\pm6 & 2.39\pm0.54 & 2.01\pm0.22 & 202\pm1 & 1.17\pm0.24 & 0.49\pm0.08 & 0.10\pm0.14 & 0.62\pm0.06 \\ 
    {\rm S0805}   & 1.01\pm0.01 & 201 & 427\pm58 & 2.22\pm0.97 & 1.71\pm0.42 & 155\pm13 & 0.53\pm0.13 & 0.17\pm0.04 & 1.03\pm0.07 & 0.62\pm0.05 \\ 
    {\rm NGC3402} & 0.96\pm0.02 & 127 & 470\pm9 & 2.95\pm0.18 & 1.35\pm0.16 & 210\pm2 & 1.32\pm0.29 & 0.49\pm0.02 & 0.17\pm0.01 & 0.66\pm0.03 \\ 
    {\rm A194 }   & 1.37\pm0.04 & 278 & 463\pm23 & 2.83\pm0.43 & 2.72\pm0.26 & 197\pm11 & 1.09\pm0.13 & 0.47\pm0.06 & 21.6\pm4.41 & 0.71\pm0.07 \\ 
    {\rm RXCJ1840.6-7709} & 1.17\pm0.04 & 112 & 488\pm12 & 3.31\pm0.24 & 1.71\pm0.17 & 219\pm5 & 1.49\pm0.10 & 0.53\pm0.04 & 0.05\pm0.01 & 1.01\pm0.06 \\ 
    {\rm WBL154^{*}} & 1.19\pm0.02 & 254 & 508\pm8 & 3.53\pm0.18 & 2.71\pm0.16 & 233\pm4 & 1.71\pm0.07 & 0.60\pm0.04 & - & 0.68\pm0.04 \\ 
    {\rm S0301  } & 1.35\pm0.03 & 348 & 497\pm47 & 3.49\pm0.86 & 3.06\pm0.44 & 230\pm8 & 1.73\pm0.32 & 0.91\pm0.05 & 0.48\pm0.03 & 0.95\pm0.08 \\ 
    {\rm NGC1132} & 1.08\pm0.01 & 206 & 490\pm9 & 3.35\pm0.19 & 2.47\pm0.06 & 215\pm2 & 1.41\pm0.04 & 0.69\pm0.01 & 0.63\pm0.03 & 0.84\pm0.04 \\ 
    {\rm IC1633 }  & 2.80\pm0.06 & 263 & 797\pm80 & 14.4\pm2.62 & 11.5\pm1.65 & 289\pm35 & 3.43\pm1.25 & 1.59\pm0.43 & 5.26\pm1.21 & 2.07\pm0.64 \\ 
    {\rm NGC4325} & 1.00\pm0.01 & 252 & 435\pm3 & 2.34\pm0.05 & 1.69\pm0.03 & 205\pm1 & 1.22\pm0.01 & 0.62\pm0.01 & 0.14\pm0.01 & 1.29\pm0.07 \\ 
    {\rm RXCJ2315.7-0222} & 1.39\pm0.02 & 425 & 552\pm25 & 4.78\pm0.68 & 3.79\pm0.15 & 253\pm10 & 2.30\pm0.30 & 1.14\pm0.09 & 0.92\pm0.44 & 1.31\pm0.10 \\ 
    {\rm NGC6338} & 1.97\pm0.09 & 436 & 671\pm39 & 8.59\pm1.50 & 6.34\pm0.33 & 295\pm15 & 3.66\pm0.13 & 1.96\pm0.17 & 0.69\pm0.11 & 2.58\pm0.40 \\ 
    {\rm IIIZw054}& 2.17\pm0.02 & 312 & 694\pm14 & 9.51\pm0.57 & 9.06\pm0.37 & 307\pm5 & 4.11\pm0.19 & 2.42\pm0.10 & 4.95\pm0.97 & 3.90\pm0.24 \\ 
    {\rm IC1262 } & 1.87\pm0.02 & 480 & 625\pm11 & 6.96\pm0.37 & 5.34\pm0.29 & 208\pm13 & 1.28\pm0.23 & 0.62\pm0.11 & 0.88\pm0.09 & 3.28\pm0.58 \\ 
    {\rm AWM4  	} & 2.45\pm0.03 & 333 & 724\pm38 & 10.8\pm1.72 & 6.50\pm0.35 & 309\pm12 & 4.19\pm0.49 & 2.16\pm0.15 & 4.41\pm0.25 & 2.87\pm0.25 \\ 
    {\rm A3390^{*}} & 1.58\pm0.06 & 272 & 543\pm21 & 4.37\pm0.51 & 1.94\pm0.19 & 253\pm14 & 2.22\pm0.37 & 0.51\pm0.07 & - & 1.41\pm0.11 \\ 
    {\rm CID28}   & 2.01\pm0.02 & 360 & 655\pm11 & 8.00\pm0.42 & 6.27\pm0.16 & 295\pm3 & 3.66\pm0.13 & 1.77\pm0.04 & 2.50\pm0.22 & 1.82\pm0.09 \\ 
    {\rm UGC03957}& 2.66\pm0.03 & 496 & 751\pm20 & 12.1\pm0.95 & 8.17\pm0.46 & 357\pm5 & 6.47\pm0.28 & 3.05\pm0.13 & 0.81\pm0.03 & 5.27\pm0.35 \\ 
    % \hline
    \end{tabular}
    \label{tab:propgroups}
    $$
    \tablefoot{The total masses for WBL154 and A3390 (marked with a star) which show a double peak have been calculated by summing up the total mass of each of their two components. The $R_{500}$ and $R_{2500}$ have been then estimated from their total $M_{500}$ and $M_{2500}$, respectively. $R_{spec}$ is the radius within which the global temperature of the object has been determined. $L_{X,xmm}$ is the luminosity in the 0.1-2.4 keV band.}
\end{table*}